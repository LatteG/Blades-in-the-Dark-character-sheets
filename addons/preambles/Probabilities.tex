\newif\ifexpectedvalue

\newcommand{\SuccessProbability}[1]{
    \ifnum#1<1
        % This calculation is only to make typesetting more consistent easier
        \pgfmathparse{100/4}
        \pgfmathprintnumber{\pgfmathresult}\%
    \else
        \pgfmathparse{(1 - 0.5^#1) * 100}
        \pgfmathprintnumber{\pgfmathresult}\%
    \fi
}

\newcommand{\ExpectedSum}[1]{
    \pgfmathparse{1^(#1) + 2^(#1) + 3^(#1) + 4^(#1) + 5^(#1)}
}

\newcommand{\ExpectedValue}[1]{
    \ifnum#1<1
        \ExpectedSum{2}
        \pgfmathparse{1 + 6^(-2) * \pgfmathresult}
    \else
        \ExpectedSum{#1}
        \pgfmathparse{6 - 6^(-#1) * \pgfmathresult}
    \fi
    E = \pgfmathprintnumber{\pgfmathresult}
}

\newcommand{\OverindulgeRisk}[2]{
    \ifnum#1<1
        \pgfmathparse{(((6-#2)/6)^2)*100}
    \else
        \pgfmathparse{(1-((#2)/6)^#1)*100}
    \fi
    \pgfmathprintnumber{\pgfmathresult}\%
}

\newcommand{\InsightCalc}{
    \pgfmathtruncatemacro{\InsightVal}{min(\Hunt, 1) 
                                     + min(\Study, 1)
                                     + min(\Survey, 1)
                                     + min(\Tinker, 1)}
}

\newcommand{\ProwessCalc}{
    \pgfmathtruncatemacro{\ProwessVal}{min(\Finesse, 1) 
                                     + min(\Prowl, 1)
                                     + min(\Skirmish, 1)
                                     + min(\Wreck, 1)}
}

\newcommand{\ResolveCalc}{
    \pgfmathtruncatemacro{\ResolveVal}{min(\Attune, 1) 
                                     + min(\Command, 1)
                                     + min(\Consort, 1)
                                     + min(\Sway, 1)}
}

\newcommand{\ViceAttributeCalc}{
    \pgfmathtruncatemacro{\ViceVal}{min(\InsightVal, min(\ProwessVal, \ResolveVal))}
}

\newcommand{\AttributeFormat}[1]{
    \textcolor{white}{\scriptsize{#1}}
}

\newcommand{\ActionFormat}[1]{
    \scriptsize{#1}
}
