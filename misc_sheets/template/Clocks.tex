\usepackage{tikz}
\usetikzlibrary{math}

\newcommand{\circlesector}[6]{
    \tikzmath{
        \arcx = #1 + cos(#3) * #5;
        \arcy = #2 + sin(#3) * #5;
    }

    \ifnum #6=1
        \fill[black] (#1,#2) -- (\arcx, \arcy) arc (#3 : #3 - #4 : #5) -- cycle;
    \fi
    \draw[black] (#1,#2) -- (\arcx, \arcy) arc (#3 : #3 - #4 : #5) -- cycle;
}

\newcommand{\clock}[5]{
    \tikzmath{
        \maxnum = #4 - 1;
        \offset = 360 / #4;
    }
    
    \foreach \num in {0, ..., \maxnum} {
        \circlesector{#1}{#2}{90 - \offset * \num}{\offset}{#3}{\ifnum\num<#5 1 \else 0 \fi}
    }
}

\newcommand{\clocks}[2]{
    \tikz{
        \tikzmath{
            \radius    = 0.35;
            \clockamnt = floor((#1 - 1) / 8);
        }
        
        \foreach\num in {0, ..., \clockamnt} {
            \tikzmath{
                \x       = \num * \radius * 2.2;
                \y       = 0;
                \segs    = min(8, #1 - \num * 8);
                \covered = #2 - max(0, #1 - \segs - \num * 8);
            }
            \clock{\x}{\y}{\radius}{\segs}{\covered}
        }
    }
}
